\chapter*{Úvod}
\addcontentsline{toc}{chapter}{Úvod}

Mluvené slovo je pro člověka nejjednodušším způsobem komunikace. Je to
způsob rychlý, efektivní a nevyžaduje použití hmatu či zraku. Tyto smysly
tak zůstávají volné pro využití jiným způsobem, například můžeme zároveň
řídit auto či vařit oběd. V případě mateřského jazyka je navíc použita
syntaxe, kterou se učíme od narození, tedy taková komunikace nevyžaduje
nějakou speciální znalost, jako seznam příkazů programu nebo funkce
ovládacích prvků.

Přirozený jazyk proto vypadá jako v mnoha směrech ideální médium pro výměnu
informací mezi uživatelem a strojem. Proč ho tedy nepoužíváme víc už dávno?
Extrahovat z něj význam tak, aby mohl být pochopen strojem, není vůbec
jednoduché. Většího průlomu v tomto směru se podařilo dosáhnout až s příchodem
technologií strojového učení, které jsou schopny se složité zákonitosti
jazyka \uv{samy} naučit. Ani to však není samospásné, pro naučení modelu,
který porozumí jazyku, je stále potřeba mnoho práce, dat a počítačového
výkonu. Pomyslně na druhém konci komunkace pak leží problém vygenerovat
odpověď zpět uživateli, který není o moc jednodušší.

Pokrok v těchto směrech dal možnost vzniku a rozšíření
\textit{hlasových asistentů}, které dnes již skoro každý nosí ve svém
chytrém telefonu a mnozí je navíc mají doma v podobě \uv{chytrého}
reproduktoru. Jak již bylo zmíněno, vývoj takové služby je náročný v mnoha
směrech, proto jsou asistenti obvykle dostupní jen v nejpoužívanějších
jazycích, jako je angličtina, španělština nebo němčina. Nakolik je autorům
známo, jediným asistentem podporujícím češtinu je \textit{IBM Watson Assistant}
(dále WA) a to pouze v textové podobě.

Cílem této práce je vytvořit hlasového asistenta v češtině, který bude schopný
odpovídat na pozdravy a především zavolat kontakty ze seznamu v telefonu.
Toho bude dosaženo propojením WA s vlastní komponentou
pro porovnání se seznamem kontaktů, \textit{Google STT/TTS}
(rozpoznání a generování mluvené řeči) a funkcionalitou telefonu. Výhodou
je relativně jednoduché rozšíření asistenta přidáním variant dialogu do WA
a ošetřením speciálních odpovědí z WA v mobilní aplikaci, aby iniciovaly
například otevření jiné aplikace nebo rozsvícení svítilny.

V příštích několika kapitolách je uvedeno shrnutí teorie související
s tématem, následuje popis vlastní implementace a zkušenosti uživatelů,
kteří aplikaci zkoušeli. V závěru je rozebrána úspěšnost splnění
vytyčených cílů.