\chapter*{Úvod}
\addcontentsline{toc}{chapter}{Úvod}

Mluvené slovo je pro člověka nejjednodušším způsobem komunikace. Je to
způsob rychlý, efektivní a nevyžaduje použití hmatu či zraku. Tyto smysly
tak zůstávají volné pro využití jiným způsobem, například můžeme zároveň
řídit auto či vařit oběd. V případě mateřského jazyka je navíc použita
syntaxe, kterou se učíme od narození, tedy taková komunikace nevyžaduje
nějakou speciální znalost, jako seznam příkazů programu nebo funkce
ovládacích prvků.

Přirozený jazyk proto vypadá jako v mnoha směrech ideální médium pro výměnu
informací mezi uživatelem a strojem. Proč ho tedy nepoužíváme víc už dávno?
Extrahovat z něj význam tak, aby mohl být pochopen strojem, není vůbec
jednoduché. Většího průlomu v tomto směru se podařilo dosáhnout až s příchodem
technologií strojového učení, které jsou schopny se složité zákonitosti
jazyka \uv{samy} naučit. Dnes pravděpodobně nejvyspělejším modelem je
ten nazvaný GPT-3, který představili \citet{brown_language_2020}.
Ani to však není samospásné, pro naučení modelu,
který porozumí jazyku, je stále potřeba mnoho práce, dat a počítačového
výkonu. Pomyslně na druhém konci komunikace pak leží problém vygenerovat
odpověď zpět uživateli, který není o moc jednodušší.

Pokrok v těchto směrech dal možnost vzniku a rozšíření
\textit{hlasových asistentů}, které dnes již skoro každý nosí ve svém
chytrém telefonu a mnozí je navíc mají doma v podobě \uv{chytrého}
reproduktoru. Grafické shrnutí důležitých milníků uvádí \citet{voicebotai_2021},
podrobněji historii popisují \citet[strany 523-524]{jurafsky_slp_2020}.

Jak již bylo zmíněno, vývoj takové služby je náročný v mnoha
směrech, proto jsou asistenti obvykle dostupní jen v nejpoužívanějších
jazycích, jako je angličtina, španělština nebo němčina. Nakolik je autorovi
známo, češtinu podporují asistenti \textit{IBM Watson Assistant}
(dále WA), ten pouze v textové podobě, a \textit{Antelli}. Ten však sice
umožňuje přidání funkcí do původní aplikace, ale ne vlastní stavbu dialogu.

Cílem této práce je vytvořit hlasového asistenta v češtině, který bude schopný
odpovídat na pozdravy a především zavolat kontakty ze seznamu v telefonu.
Toho bude dosaženo propojením WA s vlastní komponentou
pro porovnání se seznamem kontaktů, \textit{Google STT/TTS}
(rozpoznání a generování mluvené řeči) a funkcionalitou telefonu. Velkou výhodou
je relativně jednoduché rozšíření asistenta. O řízení dialogu se totiž stará WA,
takže zde můžeme přidat další pochopení uživatelových úmyslů a odpovědi na ně,
případně speciální odpovědi, na které bude vnitřně reagovat aplikace v telefonu.
Ošetřením těchto speciálních odpovědí z WA v mobilní aplikaci pak můžeme
iniciovat například otevření jiné aplikace, rozsvícení svítilny, puštění
hudby či cokoliv dalšího telefon umí.

Uvedená porovnávací komponenta má sloužit obecnějšímu účelu v rámci
kni\-ho\-vny \texttt{mConversation} firmy MAMA AI.

V kapitole~\ref{chapter-theory} shrnujeme obecnou teorii ohledně dialogových
systémů, v kapitole~\ref{chapter-wa} pak detaily o WA. Kapitola~\ref{chapter-implementation}
popisuje naši implementaci systému a nakonec kapitola~\ref{chapter-results}
obsahuje vyhodnocení úspěšnosti systému a návrhy na vylepšení
na základě sesbírané kritiky od uživatelů.