%%% A template for a simple PDF/A file like a stand-alone abstract of the thesis.

\documentclass[12pt]{report}

\usepackage[a4paper, hmargin=1in, vmargin=1in]{geometry}
\usepackage[a-2u]{pdfx}
\usepackage[utf8]{inputenc}
\usepackage[T1]{fontenc}
\usepackage{lmodern}
\usepackage{textcomp}

\begin{document}

%% Do not forget to edit abstract.xmpdata.

Dialogue Systems are getting more and more popular, among users and companies alike.
Users enjoy using smartphones and smart speakers; companies can hire fewer workers in support
centres. The main goal of this thesis is to implement a dialogue system for voice dialing in
the Czech language. The system is implemented as a mobile application for phones with
the Android operating system which utilizes Google STT/TTS for voice communication
with the user. The dialogue is handled by an instance of IBM Watson Assistant, which we
have developed for the domain. Entities found by the assistant are matched against the
user's contact list using a newly implemented matching component. This component takes
the raw textual input into account to improve on the entities recognized by Watson Assistant.
The application is implemented in the Kotlin language, the matching component in Python.
The system was evaluated with real users. 15 test users tried to make 91 phone
calls and 51 of them were successful, which means a success rate of \(56\,\rm \%\). Based on the
user feedback we came up with ideas for improvements.

\end{document}
