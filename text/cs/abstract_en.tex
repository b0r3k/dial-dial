%%% A template for a simple PDF/A file like a stand-alone abstract of the thesis.

\documentclass[12pt]{report}

\usepackage[a4paper, hmargin=1in, vmargin=1in]{geometry}
\usepackage[a-2u]{pdfx}
\usepackage[utf8]{inputenc}
\usepackage[T1]{fontenc}
\usepackage{lmodern}
\usepackage{textcomp}

\begin{document}

%% Do not forget to edit abstract.xmpdata.

Dialogue Systems are getting more and more popular, among users and companies alike.
Users enjoy using smart phones and speakers; companies can hire fewer workers in support
centres. In this thesis, we summarize the basic theory about such systems, their taxonomy
and how they are built. Then we shortly introduce the IBM Watson Assistant. The main
part is a description of our own implementation of a dialogue system for voice dialling in
the Czech language. The system is implemented as an application for mobile phones with
the Android operating system, which utilizes Google STT/TTS for voice communication
with the user. The dialogue is handled by an instance of IBM Watson Assistant, which we
have set up. The matching of the user’s contacts with his input is supplied by a matching
component that we implemented and which runs in the IBM cloud. The application is
implemented in Kotlin language, the matching component in Python language. In the
last part, we evaluate the effectiveness of our system. 15 test users tried to make 91 phone
calls and 51 of them were successful, which means a success rate of \(56\,\rm \%\). Based on the
feedback we came up with ideas for improvements, which we also present.

\end{document}
