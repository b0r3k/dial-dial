%%% A template for a simple PDF/A file like a stand-alone abstract of the thesis.

\documentclass[12pt]{report}

\usepackage[a4paper, hmargin=1in, vmargin=1in]{geometry}
\usepackage[a-2u]{pdfx}
\usepackage[utf8]{inputenc}
\usepackage[T1]{fontenc}
\usepackage{lmodern}
\usepackage{textcomp}

\begin{document}

%% Do not forget to edit abstract.xmpdata.

Dialog Systems are getting more and more popular, between users and companies alike.
Users enjoy using smart phones and speakers, companies can hire fewer workers of
support centers. In this thesis we summarize the basic theory about such systems,
their taxonomy and how they are built. Then we shortly introduce IBM Watson Assistant.
The main part is description of our own implementation of dialogue system for voice
dialing in Czech language. The system is implemented as an application for mobile
phones with Android operating system, which for voice communication with the user
utilizes Google STT/TTS. The dialog is handled by an instance of IBM Watson Assistant,
which we have set up. The matching of user's contacts with his input is supplied
by a matching component which we implemented and which runs in IBM cloud. The application
is implemented in Kotlin language, the matching component in Python language.
In the last part we evaluate effectivity of our system. 15 test users tried to make
91 phone calls and 51 of them were successful,
which is a success rate of \(56\,\rm \%\). Based on the feedback we came up
with ideas for improvements, which we also present.

\end{document}
