%%% Šablona pro jednoduchý soubor formátu PDF/A, jako třeba samostatný abstrakt práce.

\documentclass[12pt]{report}

\usepackage[a4paper, hmargin=1in, vmargin=1in]{geometry}
\usepackage[a-2u]{pdfx}
\usepackage[czech]{babel}
\usepackage[utf8]{inputenc}
\usepackage[T1]{fontenc}
\usepackage{lmodern}
\usepackage{textcomp}

\begin{document}

%% Nezapomeňte upravit abstrakt.xmpdata.

Dialogové systémy jsou dnes velmi populární, jak mezi uživateli chytrých
telefonů či reproduktorů, tak mezi firmami, které je využívají ke snížení
množství potřebných pracovníků zákaznické podpory. V této práci
shrnujeme základní dělení systémů spolu s přístupy a technikami
jejich konstrukce, následně krátce popisujeme asistenta IBM Watson.
Hlavní částí je představení vlastní implementace dialogového systému
pro hlasové vytáčení v češtině spolu s vyhodnocením jeho úspěšnosti
na základě zkušeností reálných uživatelů. Celkově se 15 uživatelů pokusilo
zahájit 91 hovorů, z čehož 51 se podařilo, což dává úspěšnost \(56\,\rm \%\).
Podle zpětné vazby jsme navrhli možná vylepšení. Systém je implementován jako aplikace
pro mobilní telefony s operačním systémem Android, která k rozpoznání
a syntéze řeči využívá služby Google. Řízení dialogu zajišťuje instance asistenta
IBM Watson, vytvořená námi za využití služeb poskytovaných IBM.
Porovnání se seznamem kontaktů provádí námi implementovaná komponenta běžící
v cloudu IBM. Aplikace je psána v jazyce Kotlin, porovnávací komponenta
v jazyce Python.

\end{document}
