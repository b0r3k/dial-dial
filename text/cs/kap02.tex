
\chapter{Konstrukce dialogových systémů}

Začneme popisem obecné terminologie, následovat bude krátké shrnutí dvou
nejběžnějších přístupů ke konstrukci. Tradiční je přístup, který bychom
mohli označit jako postupný. Taková konstrukce čítá několik komponent, kde
každá zajišťuje část práce nutné k porozumění uživateli a zodpovězení jeho
prohlášení. Ten uvedeme jako první, neboť dobře ilustruje, co všechno vlastně
člověk podvědomě při komunikaci dělá. Dále uvedeme tzv. \textit{end-to-end}
systémy využívající metod strojového učení, které umožňují nahrazení až
několika komponent jedním modelem. Tento přístup první uvedený v mnohém
předčil, ale je běžnou praxí obě konstrukce kombinovat.

\section{Terminologie}

Během dialogu se střídají promluvy dvou a více stran. Každou takovou promluvu
budeme označovat jako \textit{tah}. Při snaze zaznamenat jednodušeji význam
vysloveného obvykle používáme trojice
\textit{úmysl}--\textit{slot}--\textit{hodnota}. Pro ilustraci například
úryvek \uv{v deset hodin} bychom mohli přeložit na trojici
informovat--čas--10:00.