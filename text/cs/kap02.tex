
\chapter{Konstrukce dialogových systémů}

Krátce shrneme dva
nejběžnější přístupy ke konstrukci. Tradiční je přístup, který bychom
mohli označit jako postupný. Taková konstrukce čítá několik komponent, kde
každá zajišťuje část práce nutné k porozumění uživateli a zodpovězení jeho
prohlášení. Ten uvedeme jako první, neboť dobře ilustruje, co všechno vlastně
člověk podvědomě při komunikaci dělá. Dále uvedeme tzv. \textit{end-to-end}
systémy využívající metod strojového učení, které umožňují nahrazení až
několika komponent jedním modelem. Tento přístup první uvedený v mnohém
předčil, ale je běžnou praxí obě konstrukce kombinovat.

TODO možná terminologie? spíš ne

\section{Tradiční inkrementální systémy}

Již jsme zmínili, že takové systémy mají několik komponent, které na sebe
navazují, jedna obvykle určitým způsobem využije výstup z předchozí jako svůj
vstup.

\subsection{Převod mluvené řeči na text}

Zkráceně značíme nejčastěji \textit{STT} z anglického \textit{speech-to-text}.
Jak název napovídá, úkolem této komponenty je převést zvukový projev na text.
Dříve byly k tomuto účelu využívány především \textit{skryté
    Markovovy modely (hidden Markov models, HMM)}. Tyto modely se nazývají skryté,
protože jejich vnitřní stavy nemohou být pozorovány, vidíme pouze výstup.
Jsou postaveny na \textit{Markovových řetězcích}, jejichž základním předpokladem
je, že v sekvenci stavů pravděpodobnost příštího závisí jen na stavu aktuálním,
nikoliv žádném předchozím, jak uvádí například \citet[strana 4]{brooks_handbook_2011}.
Jejich výhodou je, že jsou poměrně jednoduché a nenáročné na výpočetní výkon.

Jako v mnoha jiných odvětvích, i zde přišly ke slovu neuronové sítě, které HMM
ve většině směrů překonaly. V dobrých podmínkách jsou schopny provést přepis
téměř bezchybně. Opět však narážíme i na jejich slabé stránky,
totiž vysokou náročnost na množství trénovacích dat a výpočetní výkon.

O obecných problémech typu okolního hluku či různé výslovnosti jsme se
již zmiňovali. Zde ještě dodáme, že komplikací může být i nedostatečná kvalita
nahrávky. Mimo jiné z těchto důvodů výstupem často nebývá jen jedno
slovo či věta, nýbrž několik spolu s \textit{jistotou}, kterou model této
variantě přiřadil.

\subsection{Extrakce významu}

Nyní když máme textovou reprezentaci výpovědi, budeme se z ní snažit nějak
jednoduššeji vyjádřit podstatné části. Této komponentě se obvykle říká
\textit{porozumění jazyku (natural language understanding, NLU)}. V jistém
smyslu je její úkol nejnáročnější, neboť jazyky mají mnoho nejasností,
víceznačností a nuancí obecně. Výstupem této komponenty je často opět
seznam reprezentací s hodnotou, nakolik si je systém tou konkrétní
interpretací jistý.

Jako onu jednodušší reprezentaci často používáme trojice
\textit{úmysl}--\textit{slot}--\textit{hodnota}. Pro ilustraci například
úryvek \uv{v deset hodin} bychom mohli přeložit na trojici
informovat--čas--10:00. K získání relevantních trojic můžeme využít ručně
psané \textit{regulární výrazy} vyhledávající vzorce v textu, tento přístup
je však dost náročný. Pro rozumnou funkcionalitu takových výrazů musíme napsat
stovky. Dnes obvykle lepší alternativou jsou opět modely využívající strojové
učení.

\subsection{Zachování stavu}

Od systému samozřejmě budeme vyžadovat určitou paměť. Pokud řekneme, že
chceme někomu zavolat, pak dostaneme otázku komu, a pak ji zodpovíme, očekáváme,
že systém si bude ještě pamatovat, že chceme volat. Tato komponenta je
značena \textit{DST} z anglického \textit{Dialogue State Tracker}. V případě
zmíněné reprezentace pomocí trojic paměti docílíme obvykle tak, že si pro každý
slot pamatujeme hodnotu. Buď jednu, kterou v případě detekce jiné ihned přepíšeme,
nebo si pro každý slot pamatujeme pravděpodobnostní rozložení více hodnot, které
průběžně upravujeme. Téma hezky shrnují Williams, Raux a Handerson ve svém článku
\citep{williams_dialog_2016}. Paměť nesmíme zapomenout v určitých případech
resetovat, například když uživatel změní celý svůj cíl, jím dříve
zmíněné hodnoty se stávají irelevantními.