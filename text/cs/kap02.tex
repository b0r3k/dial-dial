\chapter{Watson Assistant}

Jak již bylo napsáno, WA je asistent vyvíjený firmou IBM.
V této kapitole nejdříve uvedeme v sekci~\ref{wa-common} obecné informace
o práci s ním, a pak se v sekci~\ref{wa-inside} pokusíme shrnout zjištěné
informace o vnitřním fungování.
Není jich bohužel příliš mnoho, jedná se o komerční produkt,
a tak podrobnosti nejsou dostupné veřejně.

\section{Obecné informace}\label{wa-common}

\subsection{Popis}

WA je určen především pro firmy, nabízí relativně jednoduché vytvoření
vlastního modelu strojového učení na míru podle vložených trénovacích
dat. Propojením s dalšími službami umí kromě předdefinovaných odpovědí
také vyhledávat v databázi, zavolat vnější API pomocí webhooku a odpověď
z něj využít v konverzaci, nebo v případě nutnosti předat dialog lidskému
pracovníkovi. Je možné ho integrovat do existujících komunikačních platforem
jako je Facebook Messenger, Slack nebo WhatsApp, propojit s telefonní
bránou, či přidat do webové stránky nebo aplikace. Poskytuje také monitorování
úspěšnosti proběhlých dialogů \citep{wa_about}.

Přímo podporuje 13 jazyků včetně
češtiny \citep{wa_languages}, ale nabízí také univerzální model. V případě
jeho zvolení se použije předtrénovaný model chápající základní pravidla
vyskytující se ve většině jazyků, který se dotrénuje na poskytnutých
datech v daném jazyce \citep{wa_universal_model}.

\subsection{Vytvoření asistenta}

Každý vytvořený asistent musí mít nějaké \textit{dovednosti} (\textit{skills}),
což jsou prostředí s definovanými úmysly a entitami, které bude asistent
rozpoznávat, a strukturou dialogu. Více dovedností mezi sebou může komunikovat
a řízení dialogu si předávat. Entity mohou být založené buď na rozpoznávání tvarů
konkrétních slov, nebo na rozpoznávání dle vzorce -- například pro české telefonní
číslo bychom mohli definovat vzorec pro devět číslic, volitelně s mezerami.

WA využívá \textit{stromovou} reprezentaci dialogu,
kde sekvenčně kontrolujeme splnění podmínky (detekci úmyslu nebo entity) v
jednotlivých sourozencích. Jakmile v jednom najdeme podmínku splněnou, vybereme
odpověď z něj a pokud nějaké má, zanoříme se do jeho synů.

\section{Vnitřní fungování}\label{wa-inside}

Nějaké shrnutí z https://www.ibm.com/cz-en/cloud/watson-assistant a

https://www.ibm.com/blogs/watson/2020/12/watson-assistant-improves-intent-detection-accuracy-leads-against-ai-vendors-cited-in-published-study/