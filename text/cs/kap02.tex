\chapter{Watson Assistant}\label{chapter-wa}

Tato kapitola slouží k představení WA,
kterého využíváme k řízení dialogu.
Nejdříve uvedeme v sekci~\ref{wa-common} obecné informace
o práci s ním, a pak se v sekci~\ref{wa-inside} pokusíme shrnout zjištěné
informace o vnitřním fungování.

\section{Obecné informace}\label{wa-common}

\subsection{Popis}

Jak již bylo napsáno, WA je asistent vyvíjený firmou IBM.
Je určen především pro firmy, nabízí relativně jednoduché vytvoření
vlastního modelu strojového učení na míru podle vložených trénovacích
dat. Propojením s dalšími službami umí kromě předdefinovaných odpovědí
také vyhledávat v databázi, zavolat vnější API pomocí webhooku (propojení,
které v určitou chvíli pošle požadavek jinam na web) a odpověď
z něj využít v konverzaci, nebo v případě nutnosti předat dialog lidskému
pracovníkovi. Je možné ho integrovat do existujících komunikačních platforem
jako je Facebook Messenger, Slack nebo WhatsApp, propojit s telefonní
bránou či přidat do webové stránky nebo aplikace. Poskytuje také monitorování
úspěšnosti proběhlých dialogů \citep{wa_about}.

WA přímo podporuje 13 jazyků včetně
češtiny \citep{wa_languages}, ale nabízí také univerzální model. V případě
jeho zvolení se použije předtrénovaný model chápající základní pravidla
vyskytující se ve většině jazyků, který se dotrénuje na poskytnutých
datech v daném jazyce \citep{wa_universal_model}.

\subsection{Vytvoření asistenta}\label{wa-create}

Každý vytvořený asistent musí mít nějaké \textit{dovednosti (skills)},
což jsou prostředí s definovanými úmysly a entitami, které bude asistent
rozpoznávat. Více dovedností mezi sebou může komunikovat
a řízení dialogu si předávat. Entity mohou být založené buď na rozpoznávání tvarů
konkrétních slov, nebo na rozpoznávání dle regulárních výrazů -- například
pro české telefonní
číslo bychom mohli definovat výraz pro devět číslic, volitelně s
předvolbou a mezerami.

Každá dovednost musí mít také strukturu dialogu.
WA využívá \textit{stromovou} reprezentaci,
kde sekvenčně kontrolujeme splnění podmínky (detekci úmyslu nebo entity) v
jednotlivých sourozencích. Jakmile v jednom vrcholu najdeme podmínku
splněnou, vybereme
odpověď z něj a příště budeme vybírat z jeho synů, pokud nějaké má.

Podrobnější návod k vytvoření asistenta včetně komplikovanějších
technik podává \citet{akbulut_common_2020}.

\section{Vnitřní fungování}\label{wa-inside}

Nejkomplikovanější a zásadní částí WA je NLU, tedy detekce úmyslů
a entit. Ke konci roku 2020 v tom byl úspěšnější než ostatní podobné
systémy \citep{qi2021benchmarking}. Využit byl způsob testování,
který popsali \citet{arora-etal-2020-hint3}. Systémy byly natrénovány
na datech vytvořených tak, aby co nejvíce připomínala reálná data.
Následně byly testovány na vstupech reálných uživatelů nasbíraných
z běžících systémů.

WA k vysoké úspěšnosti pomáhá využití \textit{AutoML}, což je
automatické hledání nejlepších algoritmů,
parametrů a vstupních rysů \citep{noauthor_watson_2020}. Tuto techniku
spolu s aktuálně nejlepšími známými postupy shrnují \citet[duben]{He_2021}.
Zkoušení všech
možných kombinací by trvalo velmi dlouho, proto je pro urychlení
využito \textit{meta-učení (meta-learning)} \citep{hospedales2020metalearning}.
Zde to konkrétně znamená, že se jiný model dříve naučil, jak které
kombinace parametrů fungují pro
různá data, a pak na nová data aplikuje kombinace, které dříve dobře
fungovaly pro podobná data.

Dále jsou využity hluboké neuronové sítě, ale v kombinaci s tradičními
metodami strojového učení jako \textit{SVM (support vector machines)}
\citep{bosertraining}, což umožňuje
rychlejší trénink i klasifikaci \citep{potdar_watson_2021}. V neposlední
řadě je potřeba zmínit přenosové učení \textit{(transfer learning)}, které
využije model předtrénovaný
na velkém množství obecnějších dat, kterému pak stačí pro dobré výsledky
v konkrétní oblasti méně dat speciálních \citep{zhuang2020comprehensive}.

Předtím, než uživatelův vstup vůbec přijde do složitého porozumění jazyku,
je potřeba zbavit se nepřesností, které by klasifikaci výrazně ztížily.
K tomu jsou využity jazykové modely, metody strojového učení a
různé metriky vzdáleností mezi řetězci. Díky tomu WA údajně \uv{porozumí}
okolnímu kontextu a dokáže překlepy a chyby opravovat v závislosti
na něm \citep{mason_ahnouncing_2019}.