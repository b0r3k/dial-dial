\chapter*{Závěr}
\addcontentsline{toc}{chapter}{Závěr}

V kapitolách~\ref{chapter-theory}~a~\ref{chapter-wa} jsme popsali teorii nutnou
k pochopení problematiky dialogových systémů a námi využívaného WA.
V kapitole~\ref{chapter-implementation} jsme představili naši
implementaci dialogového systému pro hlasové vytáčení.
Z uživatelského hlediska se jedná o aplikaci pro telefony se
systémem Android, kterou jsme implementovali. Vnitřně využívá STT/TTS
firmy Google, námi nastavenou instanci WA a námi implementovanou porovnávací
komponentu.

Nakonec v kapitole~\ref{chapter-results} popisujeme testování
aplikace 15 uživateli, kteří se dohromady pokusili zahájit 91 hovorů,
z toho 51 úspěšně. Výsledná úspěšnost našeho systému je tedy \(56\,\rm \%\).
Na základě zpětné vazby od uživatelů a naší analýzy jsme navrhli možná
vylepšení systému, která jsme uvedli taktéž v kapitole~\ref{chapter-results}.

Zajímavým výsledkem je, že \(80\,\rm \%\) uživatelů projevilo o podobnou
aplikaci zájem. Poukazuje to na atraktivitu hlasových asistentů
a také jejich nedostatky v poskytování české lokalizace.