\chapter*{Závěr}
\addcontentsline{toc}{chapter}{Závěr}

V~kapitolách~\ref{chapter-theory}~a~\ref{chapter-wa} jsme popsali teorii nutnou
k~pochopení problematiky dialogových systémů a námi využívaného WA.
V~kapitole~\ref{chapter-implementation} jsme představili naši
implementaci dialogového systému pro hlasové vytáčení.
Nakonec v~kapitole~\ref{chapter-results} popisujeme zkušenosti
uživatelů s~naší aplíkací. Na základě zpětné vazby od nich a naší analýzy jsme navrhli možná
vylepšení systému, která jsme uvedli taktéž v~kapitole~\ref{chapter-results}.

Podařilo se nám vytvořit mobilní aplikaci pro hlasové vytáčení využívající
Google STT/TTS. Úspěšně jsme ji propojili
s~instancí WA, kterou jsme nastavili pro řízení dialogu. Ta navíc komunikuje s~námi
implementovanou porovnávací komponentou, výsledky odesíláme zpět do telefonu,
kde je případně zahájen hovor. Veškerý zdrojový kód spolu se zkompilovanou aplikací
je možné najít v repositáři platformy GitHub.%
\footnote{\url{https://github.com/b0r3k/dial-dial}}

Naši aplikaci testovalo 15 uživatelů, kteří se dohromady pokusili zahájit 91 hovorů,
z~toho 51 úspěšně. Výsledná úspěšnost našeho systému je tedy \(56\,\rm \%\).
Jako hlavní příčinu problémů jsme identifikovali špatnou detekci entit.
Naše komponenta pracovala pouze s~entitami již alespoň částečně rozpoznanými
WA, jenže obecné rozpoznávání ve WA se ukázalo jako velmi obtížné. Jako nejlepší
řešení tedy navrhujeme změnit porovnávací komponentu, aby pracovala vždy s~celým
uživatelovým vstupem a ne jen rozšiřovala již rozpoznané entity.

Pokud bychom tuto vylepšenou komponentu z~WA volali jen při detekci uživatelova
úmyslu volat, zachovali bychom jednoduchou rozšiřitelnost aplikace. Jako
druhou možnost vidíme zaměřit se čistě na hlasové vytáčení a v~takovém
případě implementovat aplikaci s~výrazně jednodušším lokálně řízeným
dialogem, kde by i porovnávání probíhalo lokálně na celém vstupu. Ideální
variantou by bylo použít i jinou implementaci STT/TTS, pak by
celá aplikace mohla běžet offline.

Zajímavým výsledkem je, že 12 z~15 testovacích uživatelů projevilo o~podobnou
aplikaci zájem. Poukazuje to na atraktivitu hlasových asistentů
a také jejich nedostatky v~poskytování české lokalizace.