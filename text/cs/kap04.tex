\chapter{Vlastní vývoj}

Na začátku této kapitoly rozebereme zvolenou platformu a programovací jazyky.
Další části budou věnovány popisu komponent programu, konkrétně aplikaci pro
mobilní telefony s operačním systémem Android, vytvoření instance WA pro řízení
dialogu (s popisem získání nejběžnějších českých jmen pro trénink) a komponenty
pro porovnání entit nalezených pomocí WA se seznamem kontaktů běžící v cloudu.

\section{Platforma a programovací jazyky}

\subsection{Platforma a jazyk mobilní aplikace}
Pro mobilní aplikace máme v dnešní době v zásadě dvě možnosti, iOS nebo Android.
Zvolili jsme Android, neboť je celkově otevřenější, autoři telefon s tímto
operačním systémem vlastní a mají tedy alespoň uživatelské zkušenosti. IBM navíc
poskytuje vývojový balíček (\textit{SDK}) pro použití WA v programovacím
jazyku Java, který je použitelný i v Androidu.

Z hlediska programovacího jazyka padla volba na Kotlin. Existují různé frameworky
schopné zkompilovat pro Android program psaný téměř v libovolném programovacím
jazyce, ale většina nakonec musí používat nějakou další překladovou vrstvu.
Kotlin je jeden z jazyků, ve kterém lze psát nativní aplikace pro Android. Je
totiž založený na Javě a dokonce umožňuje interoperabilitu s ní, což umožnilo
využití zmíněného SDK. Jeho výhoda oproti Javě je, že je celkově modernější,
psaní a čtení programů v něm je díky přehlednější syntaxi jednodušší a kratší,
navíc je doporučený samotnou firmou Google jakožto vedoucím vývoje operačního
systému Android. Další vlastností, na kterou při výběru nebyl brán tak velký
zřetel, ale nakonec se ukázala jako relativně zásadní, je podpora koprogramů
(pravděpodobně známější pod anglickým názvem \textit{coroutines}).

\subsection{Volba asistenta a jeho inicializace}
Asistentů existuje na trhu mnoho, jen jeden však aktuálně oficiálně plně
podporuje český jazyk -- Watson Assistant od IBM. Proto zde volba jednoduše
padla právě na něj.

Inicializace WA lze provést v interaktivním prostředí v prohlížeči. Toto
prostředí je však v mnohém omezené. V našem případě jsme například potřebovali
vložit stovky entit pro trénování rozpoznání. Přenést je do WA přes prohlížeč
by bylo absurdně náročné, navíc složitě replikovatelné. Sáhli jsme proto opět
po SDK, pomocí kterého můžeme WA inciovat relativně krátkým skriptem. Na volbě
konkrétního programovacího jazyka zde téměř nezáleželo (skript je krátký a běží
jen jednou pro inicializaci), zvolili jsme proto Python jakožto dobře čitelný
jazyk, se kterým máme mnoho zkušeností.

\subsection{Jazyk a prostředí běhu porovnávací komponenty}

Volba jazyka byla zde také jednoduchá, kolegové z MAMA AI totiž plánovali
komponentu také využít a požadovaným jazykem byl Python. To bylo z naší
strany naprosto v pořádku, protože k tomuto jazyku máme blízko a nepředstavoval
pro toto použití žádné výraznější limitace.

Dále bylo potřeba vyřešit, kde tato komponenta poběží. Původním nápadem bylo
zprovoznění webového serveru, který by vyřizoval požadavky na tuto komponentu.
To by však vyžadovalo další kód pro správu a především hardware, na kterém by
tento server běžel. Při hledání alternativ jsme dostali doporučení na
\textit{funkce poskytované jako služby} (\textit{function as a service, FaaS}),
které nabízí většina větších poskytovatelů cloudů. Jednoduše řečeno, na daný
server lze nahrát kód, který přijímá a vrací (obvykle) soubory formátu JSON a
běží jen ve chvíli, kdy dostane nějaký požadavek. Liší se tak od standardních
webových serverů, které musí \uv{poslouchat} neustále. Konkrétně byla vybrána
služba IBM Cloud Functions především proto, že lze spravovat ze stejného účtu jako
WA. Teoreticky může být výhoda využití stejného poskytovatele ještě v tom, že
servery budou na stejném místě a tedy výměna dat mezi nimi bude probíhat rychleji,
ale to jsme nezkoumali a tedy nemůžeme potvrdit, ani vyvrátit.

\section{Mobilní aplikace}

Pro vytvoření aplikace jsme využili integrované vývojové prostředí
\textit{Android Studio}. Původním záměrem bylo využít \textit{Visual Studio Code}
spolu s \textit{vývojovými kontejnery} za použití virtualizačního softwaru
\textit{Docker}, především díky přenositelnosti a relativní nenáročnosti
na výpočetní výkon. Ukázalo se však, že tato cesta není při vývoji aplikací
pro Android v jazyce Kotlin pod operačním systémem Windows úplně běžná a
naráží na mnoho komplikací, tedy je především pro začátek silně nevhodná.
Pro kopilaci využíváme běžně používáný \textit{gradle}.

Aplikace musela vyřešit několik věcí: získání oprávnění ke všem operacím,
získání kontaktů uživatele, vytvoření \textit{session} ve WA a odeslání
jmen kontaktů, spuštění rozpoznání hlasu, po dokončení jeho výsledek odeslat
jako zprávu do WA, po získání odpovědi spustit hlasový syntetizátor, po skončení
jeho projevu buď spustit rozpoznání hlasu a celý běh znovu, nebo zahájit
hovor. Všemi součástmi se budeme zabývat v následujících podsekcích, začneme
ale celkovou koncepcí a vzhledem.

\subsection{Celková koncepce a vzhled}

Jednotlivé komponenty (obvykle reprezentované jednou obrazovkou)
se v aplikacích
pro Android nazývají \textit{Activity}. Protože primárním způsobem interakce
s aplikací má být hlas, cílili jsme na jednoduché a sebevysvětlující grafické
rozhraní. Zvolili jsme modré tlačítko se symbolem mikrofonu na černém pozadí,
které si můžete prohlédnout na obrázku TODO a které je definováno v souboru
\texttt{activity\_main.xml}. Tímto tlačítkem je zahájen dialog (v případě
dostatečných oprávnění, jejichž získání řešíme dále).

TODO obrázek rozhraní

Protože nám stačí jedna obrazovka, je celý kód umístěn v jednom souboru
\texttt{MainActivity.kt}. Pro změnu grafického rozhraní z kódu je třeba
jejich propojení, které zajišťuje instance třídy \texttt{ActivityMainBinding}.
Toto je aktuálně oficiálně doporučený postup (na úkor dříve používaného
vyhledání pomocí \texttt{id}), ale vygenerování propojovací třídy musíme
povolit v souboru \texttt{build.gradle} na úrovni modulu.

Tato instance je využita při otevření aplikace pro \uv{nafouknutí} grafického
rozhraní a následně v kódu kdykoliv chceme k rozhraní přistupovat. To v našem
případě znamená nastavení reakce na stisk tlačítka, případně jeho deaktivace
či změna vzhledu pro indikaci probíhajích procesů.

\subsection{Získání oprávnění}

Pro svou správnou funkci potřebuje aplikace povolení nahrávat zvuk, číst
kontakty, vykonávat hovory a přistupovat k internetu. Všechna požadovaná
oprávnění musí být uvedena v souboru \texttt{AndroidManifest.xml}, první
tři jsou navíc oprávněními za času běhu, tedy o ně musíme explicitně
požádat. Oprávnění k přístupu k internetu není tak zásadní (především
protože neoperuje s uživatelovými daty), tedy je v Androidu automaticky
uděleno při instalaci aplikace.

Aktuálně oficiálně doporučeným postupem (který z toho důvodu využíváme), jak
řešit oprávnění, je nejprve o ně požádat, pokud uživatel odmítne a pokusí se
aplikaci znovu použít, zobrazit vysvětlení, k čemu jsou nutná a znovu o ně
požádat, a pokud znovu odmítne, již jen ukazovat upozornění o nedostupnosti.

Pro požádání o oprávnění potřebujeme spustit jinou aktivitu (pro získání
oprávnění existuje jedna vestavěná), vyčkat na její výsledek a na základě
něj rozhodnout o dalších krocích. K tomu využíváme opět aktuálně doporučenou
metodu \texttt{registerForActivityResult}. Pro zobrazení vysvětlení potřeby
oprávnění i upozornění o nedostupnosti využíváme třídu \texttt{AlertDialog}.

\subsection{Získání kontaktů}

Kontakty potřebujeme získat z úložiště telefonu, které se chová podobně jako
databáze. Proto dotaz na ně začneme na začátku inicializace celé služby WA
asynchronně jako koprogram. V rámci něj pak získáme jména a čísla kontaktů,
které uložíme do slovníku (v Kotlinu relizovaný pomocí \texttt{MutableMap}).

\subsection{Vytvoření session ve WA a odeslání jmen kontaktů}

Po začátku dotazu na kontakty vyvoříme instanci třídy \texttt{Assistant}
pro komunikaci s WA API, které předáme potřebné parametry a následně pošleme
požadavek pro zahájení session. V Androidu není možné posílat síťové požadavky
v hlavním vlákně (protože je potřeba udržovat ho volné pro UI), takže s výhodou
opět použijeme běh jako koprogram. Následně pomocí \texttt{await} zajistíme
kontrolu dokončení získávání kontaktů a předáme je jako \textit{kontext} do
WA session. Pokud se v průběhu něco nepovede, vyjímku odchytíme a příště
se pokusíme provést celou inicializaci znovu.

\subsection{Spuštění rozpoznávání hlasu}

Obdobně jako u získání
oprávnění potřebujeme spustit nějakou aktivitu a získat její výsledek, tedy
využijeme \texttt{registerForActivityResult}, tentokrát nejde o vestavěnou
aktivitu, ale můžeme využít parametr \texttt{Intent} (úmysl) při spuštění.
Protože ho potřebujeme používat často, vytvoříme ho jen jednou při otevření
aplikace a předáme mu potřebné parametry (český jazyk, jeden výsledek, atp.).
Když aktivita spuštěná s tímto úmyslem skončí, zkontrolujeme, že proběhla
úspěšně a pokračujeme k dalšímu kroku.

\subsection{Odeslání do WA a získání odpoědi}
Opět pomocí \texttt{await} vyčkáme nachystání WA. Pokud je vše připraveno,
pošleme v koprogramu (jedná se o síťový požadavek) text do WA a získáme odpověď.
Tuto odpověď ještě zpracujeme, pokud je na prvním řádku \texttt{[call]}, jedná
se o pokyn k hovoru, na druhém řádku najdeme jméno kontaktu a na třetím text k
přečtení. Pomocí jména kontaktu zjistíme jeho číslo, vytvoříme \texttt{Intent}
volání a nastavením proměnné \texttt{launchAgain} na \texttt{false} uložíme,
že dialog skončil. V opačném případě pošleme celou odpověď k přečtení.

\subsection{Spuštění hlasového syntetizátoru}

Převod textu na řeč budeme také potřebovat používat často, proto instanci
třídy \texttt{TextToSpeech} též vytvoříme při otevření programu. Nastavíme
ji na český jazyk a přepíšeme v ní funkci \texttt{onDone}, která je spuštěna,
když skončí přehrávání hlasu. Pokud dialog neskončil, spustíme celý běh
od rozpoznání hlasu zabalený ve funkci \texttt{launchPipeline} znovu. Pokud
skončil, jsme připraveni začít volat, vytočíme tedy číslo a ukončíme session ve
WA i celou aplikaci.

\section{Inicializace WA}