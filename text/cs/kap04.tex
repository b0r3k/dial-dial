\chapter{Vlastní vývoj}

Na začátku této kapitoly rozebereme zvolenou platformu a programovací jazyky.
Další části budou věnovány popisu komponent programu, konkrétně aplikaci pro
mobilní telefony s operačním systémem Android, vytvoření instance WA pro řízení
dialogu (s popisem získání nejběžnějších českých jmen pro trénink) a komponenty
pro porovnání entit nalezených pomocí WA se seznamem kontaktů běžící v cloudu.

\section{Platforma a programovací jazyky}

\subsection{Platforma a jazyk mobilní aplikace}
Pro mobilní aplikace máme v dnešní době v zásadě dvě možnosti, iOS nebo Android.
Zvolili jsme Android, neboť je celkově otevřenější, autoři telefon s tímto
operačním systémem vlastní a mají tedy alespoň uživatelské zkušenosti. IBM navíc
poskytuje vývojový balíček (\textit{SDK}) pro použití WA v programovacím
jazyku Java, který je použitelný i v Androidu.

Z hlediska programovacího jazyka padla volba na Kotlin. Existují různé frameworky
schopné zkompilovat pro Android program psaný téměř v libovolném programovacím
jazyce, ale většina nakonec musí používat nějakou další překladovou vrstvu.
Kotlin je jeden z jazyků, ve kterém lze psát nativní aplikace pro Android. Je
totiž založený na Javě a dokonce umožňuje interoperabilitu s ní, což umožnilo
využití zmíněného SDK. Jeho výhoda oproti Javě je, že je celkově modernější,
psaní a čtení programů v něm je díky přehlednější syntaxi jednodušší a kratší,
navíc je doporučený samotnou firmou Google jakožto vedoucím vývoje operačního
systému Android. Další vlastností, na kterou při výběru nebyl brán tak velký
zřetel, ale nakonec se ukázala jako relativně zásadní, je podpora koprogramů
(pravděpodobně známější pod anglickým názvem \textit{coroutines}).

\subsection{Volba asistenta a jeho inicializace}
Asistentů existuje na trhu mnoho, jen jeden však aktuálně oficiálně plně
podporuje český jazyk -- Watson Assistant od IBM. Proto zde volba jednoduše
padla právě na něj.

Inicializace WA lze provést v interaktivním prostředí v prohlížeči. Toto
prostředí je však v mnohém omezené. V našem případě jsme například potřebovali
vložit stovky entit pro trénování rozpoznání. Přenést je do WA přes prohlížeč
by bylo absurdně náročné, navíc složitě replikovatelné. Sáhli jsme proto opět
po SDK, pomocí kterého můžeme WA inciovat relativně krátkým skriptem. Na volbě
konkrétního programovacího jazyka zde téměř nezáleželo (skript je krátký a běží
jen jednou pro inicializaci), zvolili jsme proto Python jakožto dobře čitelný
jazyk, se kterým máme mnoho zkušeností.

\subsection{Jazyk a prostředí běhu porovnávací komponenty}

Volba jazyka byla zde také jednoduchá, kolegové z MAMA AI totiž plánovali
komponentu také využít a požadovaným jazykem byl Python. To bylo z naší
strany naprosto v pořádku, protože k tomuto jazyku máme blízko a nepředstavoval
pro toto použití žádné výraznější limitace.

Dále bylo potřeba vyřešit, kde tato komponenta poběží. Původním nápadem bylo
zprovoznění webového serveru, který by vyřizoval požadavky na tuto komponentu.
To by však vyžadovalo další kód pro správu a především hardware, na kterém by
tento server běžel. Při hledání alternativ jsme dostali doporučení na
\textit{funkce poskytované jako služby} (\textit{function as a service, FaaS}),
které nabízí většina větších poskytovatelů cloudů. Jednoduše řečeno, na daný
server lze nahrát kód, který přijímá a vrací (obvykle) soubory formátu JSON a
běží jen ve chvíli, kdy dostane nějaký požadavek. Liší se tak od standardních
webových serverů, které musí \uv{poslouchat} neustále. Konkrétně byla vybrána
služba IBM Cloud Functions především proto, že lze spravovat ze stejného účtu jako
WA. Teoreticky může být výhoda využití stejného poskytovatele ještě v tom, že
servery budou na stejném místě a tedy výměna dat mezi nimi bude probíhat rychleji,
ale to jsme nezkoumali a tedy nemůžeme potvrdit, ani vyvrátit.