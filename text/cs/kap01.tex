
\chapter{Dialogové systémy}
TODO definice

\section{Dělení}
Dialogové systémy můžeme dělit dle mnoha různých specifik. Uvedeme nějaká
základní běžně používaná, ale rozhodně nebudou všechna.

Asi vůbec
nejzákladnějším je dělení na systémy \textit{zaměřené na plnění úkolů}
(rozšířený je anglický termín \textit{task-oriented}), jejichž úkolem
je dosáhnout nějakého uživatelova cíle; a \uv{\textit{tlachací}} (anglicky
\textit{chitchat}), jejichž úkolem je jen vést s uživatelem smysluplnou
konverzaci.

\textit{Doménou} systému se rozumí téma, kterému by se měl být schopen
věnovat. Podle toho, zda to dokáže jen u jednoho, nebo u více různých,
můžeme systém nazývat \textit{jednodoménový} nebo \textit{vícedoménový}.

Dále můžeme systémy dělit podle toho, kdo řídí dialog. Dotaz či změnu tématu
může iniciovat vždy pouze uživatel a systém jen odpovídat, nebo naopak, nebo
se mohou v iniciativě střídat. Implementace posledního jmenovaného je samozřejmě
nejkomplikovanější.

Posledním dělením, které uvedeme, je podle kanálů komunikace, tedy
v jaké prvotní formě je informace mezi uživatelem a systémem předávána.
Obvykle je forma na obou stranách stejná, může se však i lišit. Mezi
nejčastější patří text, mluvená řeč a obraz (ať už statický nebo
dynamický), ale existují již i roboti schopni vyjadřovat emoce pomocí
mimiky.

My se nadále budeme zabývat systémem zaměřeným na plnění úkolů v rámci jedné
domény, který má jako vstup i výstup mluvenou řeč, protože tím námi
implementovaná aplikace je.

\section{Problémy ve zpracování dialogu}

Při trochu podrobnějším pohledu na většinu rozhovorů zjistíme, že nejsou
tak jednoduché a přímočaré, jak si je představíme. To dále ztěžuje
veškeré strojové zpracování. Některé komplikace umíme řešit jen do určité
míry, některé zatím vůbec.

TODO možná nějaká ukázka

Prvním problémem je jak vůbec zjistit, kdy uživatel dialog začal, kdy
skončil svůj \textit{tah} a očekává odpověď, a kdy skončil celý dialog.
Jako lidé většinou začátek dialogu jsme schopni rozpoznat tím, že se na
nás druhý podívá či nás osloví. Pohled u stroje s pouze zvukovým vstupem
detekovat schopni nejsme, oslovení dnešní technologie již dokáží. TODO
wakewords, HMM a teď ML, třeba https://aclanthology.org/2020.nlposs-1.9.pdf
